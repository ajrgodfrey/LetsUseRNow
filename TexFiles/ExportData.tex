

\chapter{LURN\ldots{} To Export a Dataset} 
\label{ExportData} 
 
 
This chapter assumes you've got some data to export, and I mean data; not results from analyses which is discussed in Chapter~\ref{StoreResults}. You may also wish to use this information once you know how to tabulate various numerical summaries, discussed in Chapter~\ref{NumericalEDA}. 
 
\section{Creating external files} 
 
This section is just the inverse action of importing data from an external file as discussed in Chapter~\ref{ImportData}. All the \code{read} type commands have corresponding \code{write} commands. 
 
If you read Chapter~\ref{ImportData} you'll know that my preference is to use files that are easily transferred and easily checked for their accuracy using other software. In particular, I prefer to create comma separated value (csv) files using the \Rcmd{write.csv} command. If there was a \Rclass{data.frame} called \Robject{Chickens} in my current workspace that I wanted to export for sharing with another user, I would issue the following command. 
\begin{Schunk}
\begin{Sinput}
> write.csv(Chickens, file="chickens.csv") 
\end{Sinput}
\end{Schunk}
If \R{} had created rownames for the \Rclass{data.frame}, I might not want to include them. A simple additional argument is all that is required. 
\begin{Schunk}
\begin{Sinput}
> write.csv(Chickens, file="chickens.csv", row.names=FALSE) 
\end{Sinput}
\end{Schunk}
 
If another file type is required then the help page for \Rcmd{write.csv} may be useful. This page also gives the help on the \Rcmd{write.table} command. 
 
\section{Exporting data for use in alternative software} 
 
The better statistical software options allow users to import data from a range of sources and file types. If for some (very strange) reason you need to create files in a specific format for importing into another statistics package then you will need to investigate the \Rpackage{foreign} package. 
 
 


