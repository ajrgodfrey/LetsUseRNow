

 
\chapter{이 책에 관하여} 
\label{Intro} 
  

\R{}은 통계용 프로그래밍 언어환경이다. 기존 패키지들과 달리 Graphical User Interface (GUI) 기반이 아니다보니 여타 통계 소프트웨어보다 첫 학습 진입장벽이 조금 높다. 직관에 의존해 이리저리 메뉴를 살펴보는 방식이 아니기 때문이다.
 
하지만 달리 생각해 보면, 그 원리를 제대로 이해하고 자신이 \R{을 가지고 무엇을 하고 싶은 지 뚜렷이 아는 사람들에게는 \R{의 이런 특성이 장점으로 다가올 것이다. 또한, 응용프로그램을 무작정 사용하기에 앞서 그 작동 방법과 활용방안을 제대로 이해케 해준다는 점에서 \R{은 교육용 프로그램으로 가장 제격이다. 많은 사람들이 \R{을 처음 사용할 때 자신이 무엇을 하고 싶은 지에 관한 '활용목적'과 그 목적을 어떻게 달성할 것인 지에 관한 '활용방안'을 구분하지 못해 애를 먹는다.
본서는 기본적인 데이터 분석과 조작, 데이터 시각화 등 통계 입문 수업에서 꼭 필요한 내용들을 집약적으로 전달하는 데에 그 목적이 있다. 필요에 따라 해당 주제를 뒷받침할 이론적 설명을 제시하기도 하겠지만, 본서는 모든 통계 내용을 염두에 둔 종합서는 아니다. 이 책을 참고서로 삼아도 되고 학부 수준의 통계교육을 위한 자료로 활용해도 되겠다. 필요에 따라 다양하게 사용하면 된다. 책에 추가되었으면 하는 내용이 있으면 저자에게 알려 달라. 다음 편에 추가해 보도록 하겠다.
최대한 순서에 맞춰 주제를 제시하려 했으나, 간혹 이를 따르기 어려운 내용들도 분명 있었으리라. 각 장을 시작할 때 어떤 내용들이 선행되어야 하는 지 함께 제시해 두었다. 단순히 데이터를 불러와 바로 확인해 보는 작업에서는 독자 스스로 이를 따라갈 수 있다고 미리 전제한다. 필요시 인덱스를 통해 특정 명령어들에 대한 상세 설명을 찾아볼 수도 있다. 이리저리 주제를 옮겨다니며 읽어도 상관없다는 말이다.
끝으로, 재미가 덜한 장이나 절들은 과감히 건너뛰기 바란다. 그렇게 해도 \R{}을 익히는 데에 지장이 없고 다른 장들에서 결국 \R{}을 많이 배울 것이다. 당장 중요치 않은 주제로 애써 힘들어 하며 끙끙댈 필요는 없다. 그런 주제들은 그냥 편히 남겨두고 나중에 필요할 때 다시 보면 된다. 어디 도망가지 않는다.

\section{시작하기} 
 
좋다, 아마 여러분은 \R{}을 이미 설치해 두어서 이 책을 보며 바로 사용할 준비가 되어 있을 것이다. 혹시 아직 \R{}을 설치하지 않았다면 그것부터 먼저 하라. 설치 시 어떤 옵션값들을 선택해야 할 지 잘 모르면 기본값 그대로 설치하면 된다. 필자는 \R{}이 새로 업데이트될 때마다 최신 버전을 설치한다. 그리고 별 고뇌없이 기본 세팅값을 그대로 사용한다.


 
\section{\R{} 버전} 
 
\R{}은 협력 프로젝트로 지속적인 개발이 이루어지고 있다. 보통의 경우 \R{}이 새로 출시된다 해서 업데이트를 꼭 해 줄 필요는 없다. 오히려 필자는 가능한 한 특정 버전을 쭉 사용할 것을 권한다. 필자가 \R{}을 최신 버전으로 유지하는 것은 최신 버전을 사용하고 있는 제자들을 가르쳐야 할 때가 있기도 하고, 다른 \R{} 사용자들이 제작한 패키지들 중 이따금 최신 버전에서만 작동하는 것들을 대비하기 위함이지  그 외에 다른 특별한 이유는 없다.

이 글을 쓰는 시점에서 본서는 2016년 June월 21일에 출시된 버전 3.3.1로 컴파일하였다. 다른 버전에서도 모든 코드들이 똑같이 작동할 것이다. 코드에 문제가 있으면 필자에게 알려주기 바란다. 너무 오래된 \R{} 버전에서는 몇몇 코드가 제대로 작동하지 않을 수 있다는 점을 주의하자!

