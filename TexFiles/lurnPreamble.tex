
\usepackage{makeidx}
\makeindex

%%% preamble commands for all of my documents starting February 2009

%% packages to use
	\usepackage{amssymb}
	\usepackage{amsfonts}
\usepackage{hyperref}
\usepackage{graphicx}
\usepackage{calc}
\usepackage{lscape}
\usepackage{multirow}
\usepackage[figuresright]{rotating}
%% options for the rotating package 
% figuresleft, figuresright - for two sided presentation
% clockwise, or counterclockwise (not tested)




%% my mathematical shortcuts
\newcommand{\abs}[1]{\left\vert#1\right\vert}
\newcommand{\rndbrk}[1]{\left(#1\right)}
\newcommand{\sqbrk}[1]{\left[#1\right]}
\newcommand{\incurly}[1]{\left\{#1\right\}}
\newcommand{\dfrac}[2]{\frac{\displaystyle #1}{\displaystyle #2}}
\newcommand{\dblsum}[5]{\sum_{#1}^{#2}{\sum_{#3}^{#4}{#5}}}
\newcommand{\trpsum}[7]{\sum_{#1}^{#2}{\sum_{#3}^{#4}{\sum_{#5}^{#6}{#7}}}}
\newcommand{\SumAllH}[1]{{\displaystyle \sum_{h=1}^{H}}{#1}}
\newcommand{\SumAllI}[1]{{\displaystyle \sum_{i=1}^{I}}{#1}}
\newcommand{\SumAllJ}[1]{{\displaystyle \sum_{j=1}^{J}}{#1}}
\newcommand{\SumAllK}[1]{{\displaystyle \sum_{k=1}^{K}}{#1}}
\newcommand{\SumAvailK}[1]{{\displaystyle \sum_{{\rm available}~k}}{#1}}
\newcommand{\SumComK}[1]{{\displaystyle \sum_{{\rm common}~k}}{#1}}
\newcommand{\SumPairsK}[1]{  {\displaystyle \sum_{k=1}^{K-1}}{  {\displaystyle \sum_{k^\prime =k+1}^{K} {#1} }}}
\newcommand{\SumAllM}[1]{{\displaystyle \sum_{m=1}^{M}}{#1}}
\newcommand{\SumAllN}[1]{{\displaystyle \sum_{n=1}^{N}}{#1}}
\newcommand{\SumAllR}[1]{{\displaystyle \sum_{r=1}^{R}}{#1}}

%% graph float controlling commands
\renewcommand{\topfraction}{.85}
\renewcommand{\bottomfraction}{.7}
\renewcommand{\textfraction}{.15}
\renewcommand{\floatpagefraction}{.66}
\renewcommand{\dbltopfraction}{.66}
\renewcommand{\dblfloatpagefraction}{.66}
\setcounter{topnumber}{9}
\setcounter{bottomnumber}{9}
\setcounter{totalnumber}{20}
\setcounter{dbltopnumber}{9}
%% The meanings of these parameters are described on pages 199-200, section C.9 of the LaTeX manual.
%% Are there places in your document where you could 'naturally' put a \clearpage command? If so, do: the backlog of floats is cleared after a \clearpage.

%% commands for use while drafting documents
\newcommand{\placeholder}[0]{\begin{picture}(14, 14)(0, 0) \end{picture}}
\newcommand{\NoteJob}[1]{$\clubsuit${\bf {\it Job to do: #1}}}


%% default plot sizes
\setkeys{Gin}{width=0.75\textwidth}
%% color
\usepackage{color}

\usepackage[T1]{fontenc}        % Use Type 1 encoding 


% Number only to the sections in the table of contents  
\setcounter{tocdepth}{1} 

%% BibTeX settings
\usepackage[authoryear,round]{natbib}

%%% R / System symbols
\newcommand{\R}{\textsf{R}}
\newcommand{\RHome}{R\_{}HOME}
\newcommand{\rR}{{R}}
\renewcommand{\S}{\textsf{S}}
\newcommand{\SPLUS}{\textsf{S-PLUS}}
\newcommand{\rSPLUS}{{S-PLUS}}
\newcommand{\SPSS}{\textsf{SPSS}}
\newcommand{\EXCEL}{\textsf{Excel}}
\newcommand{\ACCESS}{\textsf{Access}}
\newcommand{\SQL}{\textsf{SQL}}

\newcommand{\Rpkg}[1]{\index{#1 package@\textit{#1} package}\textit{#1}}
\newcommand{\Rpackage}[1]{\index{#1 package@\textit{#1} package}\textit{#1}}
\newcommand{\Robject}[1]{\texttt{#1}}
\newcommand{\Rclass}[1]{\index{#1 class@\textit{#1} class}\textit{#1}}
\newcommand{\Rcmd}[1]{\index{#1 function@\texttt{#1} function}\texttt{#1()}}
\newcommand{\Roperator}[1]{\texttt{#1}}
\newcommand{\Rarg}[1]{\texttt{#1}}
\newcommand{\Rlevel}[1]{\texttt{#1}}
\newcommand{\file}[1]{\hbox{\rm\texttt{#1}}}
\newcommand{\code}{\texttt}
\newcommand{\stress}[1]{\textit{#1}}
\newcommand{\stressind}[1]{\index{#1}\textit{#1}}
\newcommand{\nostressind}[1]{\index{#1}#1}
\newcommand{\booktitle}[1]{`#1'}

%%% Math symbols
\newcommand{\E}{\mathsf{E}}
\newcommand{\Var}{\mathsf{Var}}
\newcommand{\Cov}{\mathsf{Cov}}
\newcommand{\Cor}{\mathsf{Cor}}
\newcommand{\x}{\mathbf{x}}
\newcommand{\y}{\mathbf{y}}
\renewcommand{\a}{\mathbf{a}}
\newcommand{\W}{\mathbf{W}}
\newcommand{\C}{\mathbf{C}}
\renewcommand{\H}{\mathbf{H}}
\newcommand{\X}{\mathbf{X}}
\newcommand{\B}{\mathbf{B}}
\newcommand{\V}{\mathbf{V}}
\newcommand{\I}{\mathbf{I}}
\newcommand{\D}{\mathbf{D}}
\newcommand{\bS}{\mathbf{S}}
\newcommand{\N}{\mathcal{N}}
\renewcommand{\P}{\mathsf{P}}

%%% links
\hypersetup{pdftitle = {Let's Use R Now}, pdfsubject = {Book}, pdfauthor = {Jonathan Godfrey}, colorlinks = {true}, linkcolor = {blue}, citecolor = {blue}, urlcolor = {red}, hyperindex = {true}, linktocpage = {true}}

%%% captions & tables
\usepackage{longtable}


%%% texi2dvi complains that \newblock is undefined, hm...
\def\newblock{\hskip .11em plus .33em minus .07em}

%%% Exercise sections
\newcounter{exercise}[chapter]
\setcounter{exercise}{0}
\newcommand{\exercise}{\item{\stepcounter{exercise} Ex.
                       \arabic{chapter}.\arabic{exercise} }}

%% URLs
\newcommand{\curl}[1]{\begin{center} \url{#1} \end{center}}
\newcommand{\email}[1]{\href{mailto:#1}{\normalfont\texttt{#1}}}


%%% hyphenations
\hyphenation{drop-out}

%%% new bidirectional quotes need
\usepackage[utf8]{inputenc}



%to replace all figures
\usepackage{float}
 \floatstyle{ruled}
\newfloat{exhibit}{tp}{lox}[chapter]
\floatname{exhibit}{Exhibit}


%\input{LURN-KnitRStart}
