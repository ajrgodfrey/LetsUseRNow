% filename GenericSettings.Rnw 

% filename QualityControl001.Rnw 
%\setcounter{chapter}{} 
\chapter{LURN\ldots{} To Analyze Quality Control Data} 
\label{QualityControl} 
 
% filename QualityControl002.Rnw 

% filename QualityControl003.Rnw  
This chapter takes a basic look at the methods of quality control. This material is not always taught in first year statistics courses and is often one of those topics that some people think can be ignored. I guess you can tell what I think given the inclusion of this topic in this document. Like many other topics, you'll need to consult a suitable textbook for a thorough examination of the topic. Some would call quality control or quality improvement a discipline not just a topic. 
 
As quality control methods are not yet implemented in the base distribution of \R{}, you will need to obtain the \Rpackage{qcc} package from CRAN, and install it. This can be done using the menus, command lines, or using file management techniques from within your operating system. Use the examples in Chapter~\ref{Additional} to help install this package. Then use the \Rcmd{library} command to make sure the package is ready for action. 
% filename QualityControl004.Rnw 
\begin{Schunk}
\begin{Sinput}
> library(qcc) 
\end{Sinput}
\begin{Soutput}
Package 'qcc' version 2.7
\end{Soutput}
\begin{Soutput}
Type 'citation("qcc")' for citing this R package in publications.
\end{Soutput}
\end{Schunk}
% filename QualityControl005.Rnw 
I've used version 2.7 of the \Rpkg{qcc} package for the work in this chapter. Please note that other packages have been developed that will also produce control charts and other quality improvement analyses. 
 
 
\section{Data} 
 
It is common for package developers to make data available to help users test the functions of the package. The \Rpkg{qcc} package uses some example from well-regarded textbooks as well as some other sources less well-known. The context of the problems is of little value to the presentation given here so I do not intend to use much time to explain the data or the formulae needed to generate the details within each of the various charts. 

The one point that is very relevant to the presentation of the rest of this chapter is the way that data are presented or stored. The commonly-used \Rclass{data.frame} with each column being a different variable works well for many of the control charts covered here but is not the easiest one to work with for the first few control charts we investigate. The other way to present data is for each row to represent a rational subgroup, and for columns to be used for elements within each subgroup. . To get a mean of a subgroup we must use the \Rcmd{rowMeans} function for example.


\section{Monitoring a process mean} 

The most common control chart for monitoring a process mean is the x-bar chart, often denoted $\bar{x}$ chart because the control statistics being plotted is the sample mean of units within a rational subgroup of data.


 
\section{Monitoring process variance} 
 
\section{Monitoring a process using proportions} 
 
 
\section{Monitoring a process using count data} 
 
% filename QualityControl006.Rnw 
\begin{Schunk}
\begin{Sinput}
> detach(package:qcc) 
\end{Sinput}
\end{Schunk}
% filename CleanUp.Rnw 

